\section{Schlussbetrachtung}

\subsection{Fazit der Auswertung}
% Sollen wir eine Hypothese erstellen?
% Vielleicht muss man weiter zurückschauen
% Da nur beim SP500 und Euronext 100 vielleicht eher bei kleinen Firmen?
% Benötigt weitere Untersuchungen
% Problematisch wenn Daten fehlen? null, null, null etc. eventuell bessere Quellen finden

Die Auswertung ergab gemischte Ergebnisse, die sich nicht eindeutig interpretieren lassen. Anhand weiterer Analysen könnte der \enquote{Weekend Effect} tiefer untersucht werden. Auffällig war vor allem, dass der Montag überwiegend schlecht ausgefallen ist für die beiden größeren Indizes (S\&P500 und den Euronext 100). Es könnte einen Zusammenhang zwischen der Größe des Index oder auch der Größe der Unternehmen und der Ausprägung des \enquote{Weekend Effect} geben. Außerdem war die Wertentwicklung am Montag des S\&P500 im Zeitraum 1970 bis 1990 jeweils negativ. Dies könnte nahelegen, dass der Effekt vor allem in der Vergangenheit größer war. Dieser Hypothese konnte nicht nachgegangen werden, da für die anderen Indizes keine Daten in diesem Zeitraum vorlagen. Diese neuen Erkenntnisse können für die weitere Analyse verwendet werden. Mit den aktuellen Ergebnissen kann keine Aussage über die Effizienz von Märkten getroffen werden.

\subsection{Fazit zur Datenbank und Strukturierung}
% Dokumente genutzt bei der die Struktur über das Einlesen festgelegt wurde. (Außer Date, kein Schema)
% Minimale Datenstruktur für den Zweck
% Datenquelle musste leicht bereinigt werden, stellt aber optimale Daten bereit
% Durch die Nutzung von YAML sehr konfigurabel
% Schnelle Datenaufbereitung dank pandas

% schnell eingearbeitet dank der SQL Sprache und der Bibliothek PyOrient
% Technisches Debugging der SQL Befehle über das Web Interface
% Das Einlesen der Daten mit Mix aus JSON und SQL funktionierte sehr gut, leider gibt es keine batch insert funktion

% SB Tree Index
% clevere Nutzung von Clustern (gute Einschränkung der Suchen nach einem konkreten Index)

Durch die Nutzung der OrientDB NoSQL Datenbank konnte auf die Erstellung eines Schemas verzichtet werden. Trotzdem erlaubt die Datenbank im vorhinein die Definition von einzelnen Eigenschaften. Mit der \texttt{Date} Eigenschaft konnte der Datentyp im vorhinein korrekt festgelegt und von einem Index erfasst werden. Die weitere Datenstruktur ergab sich während der Entwicklung aus den benötigten Eigenschaften für die Analyse. Hier konnte Administrationsaufwand eingespart werden. Durch die Nutzung der Bibliothek \enquote{pandas} und eingebauten Funktionen für das Lesen von \gls{CSV}-Dateien und die Manipulation der Daten konnte das Auslesen und die Aufbereiten optimiert werden. Die unvollständigen Datensätze aus der Quelle \enquote{yahoo Finance} konnten leicht entfernt werden. Außerdem konnte durch die zusätzliche Nutzung von \gls{YAML}-Konfigurationsdateien der Quellcode sauber gehalten und das Einlesen von Dateien automatisiert werden. Durch die Unterstützung von \gls{SQL} macht OrientDB den Umstieg sehr leicht. Zusammen mit der Bibliothek PyOrient kann die Datenbank komplett aus Python gesteuert werden. Mithilfe des Web Interfaces konnten zudem \gls{SQL} Befehle getestet und der Output schnell überprüft werden. Die zusätzliche Integration von \gls{JSON} in die Datenbanksprache erlaubte ein einfaches Einpflegen der Daten. Außerdem konnten Cluster effizient genutzt werden, um eine physisch getrennte Aufteilung der Datensätze nach ihrer \gls{ISIN} zu erlauben und damit Abfragen performanter zu machen. Zum Zweck der Analyse konnten technischen Fähigkeiten der Datenbank gut genutzt werden. Bei größeren Datenmengen wäre eine performantere Massenerstellung von Datensätzen über das \gls{JSON}-Format nützlich.

% Fehlende Massen Inserts?

\clearpage