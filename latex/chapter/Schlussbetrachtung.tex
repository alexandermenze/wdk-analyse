\section{Schlussbetrachtung}

\subsection{Fazit der Auswertung}
% Sollen wir eine Hypothese erstellen?
% Vielleicht muss man weiter zurückschauen
% Da nur beim SP500 und Euronext 100 vielleicht eher bei kleinen Firmen?
% Benötigt weitere Untersuchungen
% Problematisch wenn Daten fehlen? null, null, null etc. eventuell bessere Quellen finden

Die Auswertung ergab gemischte Ergebnisse, die sich nicht eindeutig interpretieren lassen. Anhand weiterer Analysen könnte der \enquote{Weekend Effect} tiefer untersucht werden. Auffällig war vor allem, dass der Montag überwiegend schlecht ausgefallen ist für die beiden größeren Indizes (S\&P500 und den Euronext 100). Es könnte einen Zusammenhang zwischen der Größe des Index oder auch der Größe der Unternehmen und der Ausprägung des \enquote{Weekend Effect} geben. Außerdem war die Wertentwicklung am Montag des S\&P500 im Zeitraum 1970 bis 1990 jeweils negativ. Dies könnte nahelegen, dass der Effekt vor allem in der Vergangenheit größer war. Dieser Hypothese konnte nicht nachgegangen werden, da für die anderen Indizes keine Daten in diesem Zeitraum vorlagen. Diese neuen Erkenntnisse können für die weitere Analyse verwendet werden.

\subsection{Fazit zur Datenbank und Strukturierung}
% Technische Möglichkeiten der Datenbank im Bezug auf die Datensätze bewerten
% Nutzung von Clustern
% Nutzung von Indizes
% Bewertung der gewählten Datenstruktur
% Minimal für den Zweck
% Dokumente genutzt bei der die Struktur über das Einlesen festgelegt wurde. (Außer Date)


\clearpage