\section{Schlussbetrachtung}

\subsection{Fazit}
% Markenbewusstsein (Logo, Farbe)
% RecyclerViews
% Placeholders
% Gute Speicherauslastung
% App skaliert auf allen Geräten (DPI und Skalierung der Bilder 70%)

Durch die Entwicklung eines Prototypen konnte wertvolles Feedback für die Weiterentwicklung der App gesammelt werden. Die zentralen Aspekte erhielten überwiegend positive Rückmeldungen und die inhaltlichen Vorstellungen konnten gestärkt werden. Verbesserungsvorschläge können für die Optimierung der App genutzt werden. Während der Entwicklung wurde mit Bibliotheken wie AndroidX, Glide und Gson auf stabile Industrielösungen gesetzt und gleichzeitig Aufwand eingespart. Optimierte Ressourcen und die durchgehende Nutzung von \gls{DPI} ermöglichen die Unterstützung von mehreren Geräteklassen. ViewModels und DataBinding wurden genutzt, um Benutzeroberfläche und Logik sauber zu trennen. Außerdem wurde viel Wert auf die Benutzererfahrung gelegt. Dazu zählen die Nutzung von leistungsoptimierten RecyclerViews, angepassten Mediengrößen und Platzhaltern. Um eine flüssige Integration in das Android-System zu bieten, wurde sich für eine native Anwendung entschieden und stets die Nutzung von typischen Komponenten wie die Top app und Bottom Navigation Bar bevorzugt. Außerdem wurden für Android empfohlene Material Design Guidelines eingehalten. Die intuitive Bedienung konnte durch das Feedback bestätigt werden.

\subsection{Ausblick}
% Begründet die Freigabe den Endnutzer erlauben

% Einbau einer kompletten Planungssoftware einer Garnitur
% (wie DeinKonfigurator) mit Preis
% Weitere Optimierungen durch weiteres Feedback (Beispiel: Berzüge -> Untere Leiste schwer zu treffen)
% Massentest im Unternehmen
% Direkt Bestellungen absetzen

% Authentifizierung um später Sortimente für Kunden anzuzeigen
% MediaPlayer Fullscreen

Das Feedback wird in die zukünftige Entwicklung des Prototypen mit einfließen. Eine Optimierung der Galerie und Modellsuche sowie eine bessere Integration der Bezugsdaten ist bereits geplant. Um mehr Feedback über den Inhalt zu sammeln, können jetzt Tests mit der Zielgruppe durchführt werden. Da das verfügbare Sortiment für jedes Möbelhaus individuell ist, wird in der Zukunft eine Authentifizierung in der Anwendung benötigt. Dann sollen die Verkäufer nach dem Login ihr lokales Sortiment in der App betrachten können. Zudem könnte die App in Zukunft für Endkunden zugänglich gemacht werden. Für diese könnte eine Augmented Reality Ansicht eingebaut werden. Es wäre dann möglich, Garnituren virtuell im Wohnzimmer zu platzieren, um eine digitale Vorauswahl zu treffen.

\clearpage