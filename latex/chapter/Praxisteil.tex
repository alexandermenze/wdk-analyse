\section{Praxisteil}
% CRISP DM
% Hier noch mal beschreiben das lediglich der Effekt nachgewiesen werden soll, nicht belegt
% Im Zusammenhang mit Nutzung von Indizes erläutern (Grundlage rechtfertigen)

% Optimierung: Jeden Index in ein eigenes Cluster, dann Index weg

\subsection{Problemverständnis}
% Die konkreten Berechnungen hier einfügen
% Welche Daten werden benötigt?
% Welche Werkzeuge sollen genutzt werden (Wie sollen die Daten angezeigt werden um das Problem zu erläutern?)
% Mengen und Frequenzen von Daten

% anhand von großen Indizes um die Marktentwicklung insgesamt zu betrachten und nicht auf einzelne Aktien zu sehen
% In verschiedenen Ländern und Regionen (ob es Unterschiede gibt)
% Dazu werden die Daten historisch analysiert (soweit wie diese zurückgehen)
% Benötigt werden die täglichen Kursentwicklungen
% Berechnet wird die Kursentwicklung eines Tages mit der Formel (Schlusskurs Tag - Schlusskurs letzter Tag)
% Das ganze kann dann addiert werden nach Wochentagen um eine Gesamtstatistik zu erhalten

\subsection{Datenverständnis}
% Datenstruktur und Format, Menge, Zeitauflösung etc.
% Was ist die ISIN?
% Wie und warum habe ich aufgebaut
% Adj_Close vs Close

\subsection{Datenvorbereitung}
% Einlesen der Daten und Aufbereitung
%   Behandeln von Ausreißern und fehlenden Daten zwischen den Datensätzen
%   Wie geht man mit unterschiedlicher Anzahl Wochentagen um? (Normalisierung)
% Format der Dokumente und Wahl der Tabellen
% ISIN, Region?, Institution?
% CSV Format
% Schreiben der Daten in die Datenbank
% Erstellung des Datenbankindex
% Konfiguration

\subsection{Auswertung der Daten}
% Laden der Daten aus der Datenbank
% Generierung der Auswertung mit Verweis auf das Problemverständnis
% Erstellung der Grafiken
% Zahlenoutput der Auswertung

\clearpage