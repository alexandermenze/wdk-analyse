\section{Praxisteil}
% CRISP DM
% Hier noch mal beschreiben das lediglich der Effekt nachgewiesen werden soll, nicht belegt
% Im Zusammenhang mit Nutzung von Indizes erläutern (Grundlage rechtfertigen)

% Optimierung: Jeden Index in ein eigenes Cluster, dann Index weg

Im Praxisteil dieser Arbeit soll zuerst das Problem beschrieben werden und wie es fachlich gelöst werden kann. Dazu gehört auch die Auswahl der Werkzeuge für die Umsetzung. Anschließend folgt die Strukturierung und Planung der Daten. Dies beinhaltet die Erstellung einer begründeten Datenstruktur. In dem Kapitel der \nameref{sec:datenvorbereitung} geht es dann um die Beschaffung, Aufbereitung und das Einlesen der Quelldaten in die Datenbank. Abschließend wird die Auswertung durchgeführt.

\subsection{Problemverständnis und Planung}
% Die konkreten Berechnungen hier einfügen
% Welche Daten werden benötigt?
% Welche Werkzeuge sollen genutzt werden (Wie sollen die Daten angezeigt werden um das Problem zu erläutern?)
% Mengen und Frequenzen von Daten

% anhand von großen Indizes um die Marktentwicklung insgesamt zu betrachten und nicht auf einzelne Aktien zu sehen
% In verschiedenen Ländern und Regionen (ob es Unterschiede gibt)
% Das soll über mehrere Zeiträume analysiert werden um wirklich eine Aussage treffen zu können
% Dazu werden die Daten historisch analysiert (soweit wie diese zurückgehen)
% Benötigt werden die täglichen Kursentwicklungen
% Berechnet wird die Kursentwicklung eines Tages mit der Formel (Schlusskurs Tag - Schlusskurs letzter Tag)
% Das ganze kann dann addiert werden nach Wochentagen um eine Gesamtstatistik zu erhalten

In dieser Auswertung soll untersucht werden ob der Weekend Effect in verschiedenen Regionen aufgetreten ist. Falls er existiert, ist ebenfalls interessant ob der Effekt je nach Zeitraum in der Vergangenheit unterschiedlich ausgeprägt war. Dafür werden historische Daten von yahoo Finance mit der Programmiersprache Python und der Datenbank OrientDB ausgewertet. Um bei der Auswertung möglichst die gesamte Entwicklung eines Marktes einzufangen und nicht Anomalien in den Kursverläufen einzelner Aktien zu untersuchen werden dafür ausschließlich Daten von breiter aufgestellten Aktienindizes verwendet. Benötigt werden dazu jeweils tagesaktuelle Daten

% Bibliotheken

\subsection{Datenverständnis}
% Datenstruktur und Format, Menge, Zeitauflösung etc.
% Was ist die ISIN?
% Wie und warum habe ich aufgebaut
% Adj_Close vs Close

% Daten kommen von yahoo Finance

\subsection{Datenvorbereitung}
\label{sec:datenvorbereitung}
% Einlesen der Daten und Aufbereitung
%   Behandeln von Ausreißern und fehlenden Daten zwischen den Datensätzen
%   Wie geht man mit unterschiedlicher Anzahl Wochentagen um? (Normalisierung)
% Format der Dokumente und Wahl der Tabellen
% ISIN, Region?, Institution?
% CSV Format
% Schreiben der Daten in die Datenbank
% Erstellung des Datenbankindex
% Konfiguration

\subsection{Auswertung der Daten}
% Laden der Daten aus der Datenbank
% Generierung der Auswertung mit Verweis auf das Problemverständnis
% Erstellung der Grafiken
% Zahlenoutput der Auswertung

\clearpage