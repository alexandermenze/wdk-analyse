\section{Grundlagen}

\subsection{Aktienmarkt}

\subsubsection{Aktien}
% Erklärung von Aktienmarkt Indizes
% https://www.springerprofessional.de/equities-and-equity-indices/16861122?searchResult=2.equity%20indices&searchBackButton=true&fulltextView=true
% https://www.springerprofessional.de/geld-und-vermoegensanlage/19054466?searchResult=20.equity%20index&searchBackButton=true
% https://www.springerprofessional.de/aktien/10364288 - Aktien Angebot und Nachfrage

Aktien sind Unternehmensanteile. Investoren können sich über Aktien finanziell an Unternehmen beteiligen. Die öffentlichen Anteile können an einer Börse gehandelt werden. Mit dem Besitz einer Aktie bekommt der Aktionär ein Stimmrecht auf der Hauptversammlung und wird an Gewinnausschüttungen beteiligt. Der Preis oder auch Kurs einer Aktie schwankt in der Regel während der Öffnungszeiten der Börse. Er setzt sich aus Angebot und Nachfrage zusammen. Wichtige Werte sind die Tagesöffnungs- und Tagesschlusskurse. Diese geben den Kurs jeweils zur Öffnung und zum Schluss eines Tages an einer Börse an.

\subsubsection{Aktienindizes}
% Nicht wie Aktien in Euro sondern in Punkten
% Warenkörbe von Aktien
% Bilden einen Markt oder eine Branche ab
Aktienindizes sind

\subsubsection{Kapitalmarktanomalien}
% Weekend Effect
% Anomalie, reicht zu sagen das sie existieren könnte das aber nicht bewiesen ist?

\subsection{NoSQL}

\subsection{OrientDB}

\subsubsection{Unterstützte Datenmodelle}

\subsubsection{Klassen und Schemata}

\subsubsection{Datenabfragesprache}

\subsubsection{Web Interface}

\subsubsection{Python Bibliothek}

\subsection{Datenquelle für Börsendaten}
% Wie sehen die Daten aus, wo kommen sie her und wo werden sei gespeichert?
% Fehlen der Volumina für sehr alte Daten (aber eigentlich egal)

\clearpage