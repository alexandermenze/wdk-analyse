\section{Grundlagen}

\subsection{Aktienmarkt}

\subsubsection{Aktien}
% Erklärung von Aktienmarkt Indizes
% https://www.springerprofessional.de/equities-and-equity-indices/16861122?searchResult=2.equity%20indices&searchBackButton=true&fulltextView=true
% https://www.springerprofessional.de/geld-und-vermoegensanlage/19054466?searchResult=20.equity%20index&searchBackButton=true
% https://www.springerprofessional.de/aktien/10364288 - Aktien Angebot und Nachfrage

% Preis wird in Euro / Dollar angegeben

Aktien sind Unternehmensanteile. Investoren können sich über Aktien finanziell an Unternehmen beteiligen. Die öffentlichen Anteile können an einer Börse gehandelt werden. Mit dem Besitz einer Aktie bekommt der Aktionär ein Stimmrecht auf der Hauptversammlung und wird an Gewinnausschüttungen beteiligt. Der Preis oder auch Kurs einer Aktie schwankt in der Regel während der Öffnungszeiten der Börse. Er setzt sich aus Angebot und Nachfrage zusammen.\footnote{\cite[Vgl.][35]{Chan2019}}

% Tagesschlusskurs != nächster Tag Öffnungskurs

\subsubsection{Aktienindizes}
% DAX Berechnung alle 15 Sekunden - https://www.springerprofessional.de/equities-and-equity-indices/16861122?searchResult=2.equity%20indices&searchBackButton=true&fulltextView=true 
% DAX erfüllt Informationsfunktion - https://www.springerprofessional.de/geld-und-vermoegensanlage/19054466?searchResult=20.equity%20index&searchBackButton=true

% Nicht wie Aktien in Euro sondern in Punkten
% Warenkörbe von Aktien
% Bilden einen Markt oder eine Branche ab

Ein Aktienindex gibt die Wertentwicklung einer ganzen Gruppe von Aktien wider. Sein Wert wird, im Gegensatz zu Aktien, nicht in einer Währung wie Euro oder \gls{USD} angegeben, sondern in Punkten. Diese Punkte werden aus den Preisen der enthaltenen Werte und einer, je nach Art des Index unterschiedlichen, Gewichtung ausgerechnet.\footnote{\cite[Vgl.][61]{Daume2021}} Daher kann in einen Index nicht direkt investiert werden. Er stellt lediglich eine Kenngröße für die Entwicklung einer Gruppe von Aktien dar. Für Anleger bietet ein Aktienindex Informationen über den allgemeinen Trend in einem Markt. Der \gls{DAX} vereint \gls{ZB} die Entwicklung der 30 umsatzstärksten Aktiengesellschaften Deutschlands. Er stellt damit eine übergreifende Kennziffer für die Entwicklung des Aktienmarkts in Deutschland. Andere Indizes decken ganze Kontinente oder einzelne Branchen ab. Da sich die unterliegenden Werte eines Index laufend während der Öffnungszeiten der Börse verändern, wird auch der Indexwert meistens im Laufe des Tages angepasst.\footnote{\cite[Vgl.][110]{Grundmann2021}} Für den \gls{DAX} geschieht das alle 15 Sekunden.\footnote{\cite[Vgl.][63]{Daume2021}}

\subsubsection{Kapitalmarktanomalien}
% https://www.springerprofessional.de/effiziente-maerkte/16235650?searchResult=3.kapitalmarktanomalie&searchBackButton=true
% Externer Faktor Wetter - http://arno.uvt.nl/show.cgi?fid=129516

% Weekend Effect
% Anomalie, reicht zu sagen das sie existieren könnte das aber nicht bewiesen ist?

% Fama Quelle

Die Effizienzmarkthypothese von Fama beschreibt, dass der Kapitalmarkt effizient ist, wenn alle relevanten Informationen allen Marktteilnehmern sofort zur Verfügung stehen und somit das Angebot und die Nachfrage der Marktteilnehmer direkt beeinflussen. Alle Marktteilnehmer handeln nach dieser Hypothese rational. Effizient bedeutet dann, dass bei der Veröffentlichung neuer Informationen über ein Unternehmen sich die Preise einer Aktie sofort anpassen um den korrekten Wert widerzuspiegeln.\footnote{\cite[Vgl.][9\psq]{Wagner2019}} Es gibt viele Untersuchungen sowohl für als auch gegen diese Theorie. Dagegen spricht \gls{ZB} das neben externen Faktoren, wie \gls{ZB} dem Wetter, auch periodische Anomalien einen Einfluss auf die Aktienkurse haben sollen. Dazu gehörten jährlich, saisonal, und wöchentlich auftretende Schwankungen.\footnote{\cite[Vgl.][5\psq]{Coenen2020}} Im Rahmen dieser Arbeit soll sich vor allem auf die Hypothese des \enquote{Weekend Effect} konzentriert werden. Diese besagt, dass die Kurse am Montag im Vergleich zu den anderen Wochentagen unterdurchschnittlich ausfallen und damit eine wiederkehrende Anomalie existiert. Zu dieser Erscheinung gibt es viele Untersuchungen und Vermutungen aber keine wissenschaftliche Erklärung.\footnote{\cite[Vgl.][8]{Coenen2020}}

% Erwähnen warum der Weekend Effect auftreten könnte?
% Weekend effect tritt angeblich nicht bei Feiertagen auf

\subsection{NoSQL}
% https://www.springerprofessional.de/nosql-databases/19181680?searchResult=1.nosql&searchBackButton=true&fulltextView=true
% https://www.springerprofessional.de/a-comparative-study-of-nosql-databases/18759118?searchResult=10.nosql&searchBackButton=true&fulltextView=true
% https://www.springerprofessional.de/data-integrity-verification-in-column-oriented-nosql-databases/15930056?searchResult=5.column%20oriented&searchBackButton=true&fulltextView=true

% Entwicklung von relationalen Datenbanken früher für alles (waren auch praktisch für shop Systeme mit transaktionaler Sicherheit, single source of truth)
% Massensysteme die weltweit verteilt sind haben Probleme aufgezeigt (Skalierung und Performance wegen ACID)
% Ermöglichen eine horizontale Skalierung (Auf mehrere Systeme)
% Gerade bei Big Data und Massendaten die unterschiedlich strukturiert sind ist es schwierig ein gemeinsames Datenmodell zu finden
% Typen: Key-Value Datastore, Column-oriented Data Store, Document Data Store, Graph Data Store

Angetrieben durch die hohen Anforderungen an Skalierung und Performance entwickelten sich in den späten 80er Jahren die sogenannten NoSQL Datenbanken. Markant ist vor allem die Nutzung eines nicht-relationalen Datenbankmodells. Bei diesen muss im Vorhinein kein festes Schema festgelegt werden, wodurch die Administrationsaufgaben minimiert werden. Außerdem unterstützen viele dieser Datenbanken eine Replikation auf mehrere kleine Instanzen und damit auch eine erhöhte Ausfallsicherheit und verbesserte Lastenverteilung. Um diese Ziele zu erreichen verzichten NoSQL Datenbanken häufig auf die transaktionale Sicherheit die bei relationalen Datenbanken üblich ist. Da dies nicht die optimale Lösung für jedes System ist, stellen NoSQL Datenbanken keinen Ersatz oder eine Verbesserung, sondern eine Alternative für spezielle Anwendungsfälle, dar.\footnote{\cite[Vgl.][51\psq]{Bagga2021}}

Die vier am häufigsten vertretenen Datenmodelle sind \enquote{Key-Value}, \enquote{Column-oriented}, \enquote{Document} und \enquote{Graph}. Der Key-Value Speicher funktioniert ähnlich wie ein Hash-Table. Jedem Eintrag wird ein eindeutiger Schlüssel zugeteilt. Dabei kann der Wert dahinter ein \gls{JSON} Dokument sein. Bei dem Column-oriented Datenmodell werden die Daten in Tabellen gespeichert, jedoch in einem anderen technischen Format als bei einer relationalen Datenbank. Dieses soll vor allem die Lese-Zugriffe auf den Speicher deutlich reduzieren. In den dokumentenbasierten NoSQL Datenbanken können unstrukturierte Daten ohne festes Schema gespeichert werden. Jeder Datensatz ist dabei ein unabhängiges Dokument. Bei Graph-Datenbanken besteht das Datenmodell aus Eckpunkten die über Kanten miteinander verbunden werden. Das erlaubt es große verknüpfte Graph-Strukturen darzustellen.\footnote{\cite[Vgl.][52\psq]{Bagga2021}}

% Eventuell ACID erklären

\subsection{OrientDB}
% https://en.wikipedia.org/wiki/OrientDB
% http://orientdb.com/docs/2.2.x/Transactions.html

% Schema less, mixed, full
% erlaubt erweitertes SQL als Abfragesprache
% verschiede indexing methoden
% open source (and commercial version)
% fehlertoleranz, clustering, sharding, horizontal scaling, replication
% fully ACID
% sehr schmal (nur 512 mb ram)
% encrypting stored content
% 3rd party plugins

OrientDB ist eine open-source NoSQL Datenbank die im Jahr 2010 erschienen ist. Sie wurde in Java geschrieben und ist damit komplett plattformunabhängig. Als Datenmodell können sowohl Dokumente als auch Graphen verwendet werden. OrientDB erlaubt bei der Nutzung von Dokumenten sowohl schemalose als auch schemabehaftete Modi. Des Weiteren unterstützt sie die typischen Fähigkeiten einer NoSQL wie Fehlertoleranz, Clustering, einfache horizontale Skalierung und Datenreplikation über mehrere Instanzen.\footnote{\cite[Vgl.][]{OrientDBGettingStarted}} Dabei soll die Datenbank wenig Arbeitsspeicher nutzen und erlaubt zudem Sicherheitsfeatures wie die Verschlüsselung der Daten auf der Festplatte. Im Gegensatz zu anderen NoSQL Datenbanken kann OrientDB eine transaktionale Sicherheit über mehrere Instanzen gewährleisten.\footnote{\cite[Vgl.][]{OrientDBTransactions}} Dabei unterstützt sie alle gängigen Datenformate wie Zahlen, Daten, Zeichenketten und zusätzlich Verknüpfungen (sogenannte Links) zwischen Objekten.\footnote{\cite[Vgl.][]{OrientDBTypes}}

\subsubsection{Klassen und Datenbankindizes}
% Erlaubt die Nutzung von Klassen (wie Tabelle in einer relationalen Datenbank)
% Ermöglichen die Definition von Feldern
% Können Felder haben oder Schemalos sein. Können auch aus mehreren Klassen sich zusammensetzen

Klassen in OrientDB sind Strukturinformationen über die einzupflegenden Daten. Sie können mit Tabellen in relationalen Datenbanken verglichen werden. Datensätze in der Datenbank müssen zwingend in eine Klasse eingeordnet werden. Je nach gewählter Strukturierung kann den Klassen ein Schema in Form von Eigenschaften und Datentypen zugeordnet werden. Dieses Schema ist optional und muss nicht zwingend alle Eigenschaften erfassen. Dadurch können Daten strukturiert, halb-strukturiert oder auch unstrukturiert gespeichert werden.\footnote{\cite[Vgl.][]{OrientDBClasses}} Nachdem eine Klasse erstellt wurde, können auf den Eigenschaften einer Klasse Datenbankindizes erstellt werden. In der Dokumentation werden unter anderem die folgenden drei Arten von Indizes beschrieben\footnote{\cite[Vgl.][]{OrientDBIndexes}}:

\begin{itemize}
    \item SB-Tree Index: Standardtyp der eine gute Performance liefert und auch bei \gls{SQL} Bereichssuchen genutzt werden kann.
    \item Hash Index: Bietet einen sehr schnellen Zugriff bei punktuellen Abfragen einzelner Datensätze. Kann nicht mit Bereichssuchen genutzt werden.
    \item Auto Sharding Index: Nutzt intern mehrere Hash Indizes um die Abfrage zu parallelisieren.
\end{itemize}

% Ist das Zitat für die Liste richtig?

\subsubsection{Cluster}
% http://orientdb.com/docs/2.2.x/Distributed-Architecture.html
% http://orientdb.com/docs/2.2.x/Tutorial-Clusters.html

% Jede Klasse bekommt zwangsweise ein Cluster
% Man kann konkret nach Clustern filtern
% Cluster sind physisch getrennte Einheiten
% Können auch auf unterschiedlichen Systemen liegen
% Optimierung: Jeden Index in ein eigenes Cluster, dann Index weg

Jede Klasse in der Datenbank erhält mindestens ein Cluster. Cluster sind physisch getrennte Einheiten. Das können mehrere Speicherorte auf einer Festplatte aber auch komplett getrennte Systeme sein die zusammenarbeiten. Da verschiedene Cluster parallel bearbeitet oder durchsucht werden können, kann durch die Nutzung dank Parallelisierung die Leistung der Datenbank gesteigert werden. Cluster können vom System verwaltet oder auch manuell angelegt werden. Wenn die Cluster intelligent gewählt werden, kann damit die Suche nach Daten weiter optimiert werden. Beispielsweise könnten Rechnungen nach ihrem Erstellungsjahr jeweils in ein eigenes Cluster einsortiert werden. Wenn dann nach Rechnungen in einem bestimmten Zeitraum gesucht wird, kann die Datenmenge als erstes über das Jahr stark eingeschränkt werden. Dies kann performanter als die Nutzung eines Index sein. Bei weltweit verteilten Anwendungen könnte ein Cluster für jedes Land erstellt werden, das das jeweils vom geografisch nächsten Serverstandort des Nutzer erreichbar ist.\footnote{\cite[Vgl.][]{OrientDBClusters}}

Cluster spielen auch in der Ausfallsicherheit eine wichtige Rolle. Denn sie können von mehreren Servern, die sich untereinander synchronisieren, gleichzeitig zur Verfügung gestellt werden.\footnote{\cite[Vgl.][]{OrientDBDistributedArchitecture}}

\subsubsection{Datenabfragesprache}
% http://orientdb.com/docs/2.2.x/Commands.html
% http://orientdb.com/docs/2.2.x/SQL.html

% SQL als Basis Syntax da dies sehr bekannt und oft genutzt ist
% Ohne Joins dafür mit neuen Komponenten
% Relationships durch LINKS dargestellt
% Einfache Abfrage verschachtelter Eigenschaften mit dot notation
% Select, Insert, Update, Delete

Statt eine neue Abfragesprache zu entwerfen nutzt OrientDB die aus relationalen Datenbanksystemen bekannte \gls{SQL}. Darunter die Standardschlüsselworte \texttt{SELECT}, \texttt{INSERT}, \texttt{UPDATE}, \texttt{DELETE}. Die Syntax wird dabei erweitert.\footnote{\cite[Vgl.][]{OrientDBSQL}} So können über die Kommandos \texttt{CREATE CLASS} und \texttt{ALTER CLASS} Klassen erstellt und bearbeitet werden. Außerdem kann den Klassen über die \texttt{PROPERTY} Befehle ein Schema vorgegeben werden, bei dem auch die konkreten Datentypen definiert werden können. Zudem führt die Datenbank eine komplett neue Reihe an Befehlen für Eckpunkte (\texttt{VERTICES}) und Kanten (\texttt{EDGES}) ein um die Arbeit mit Graphen zu ermöglichen. Auch können die Befehle für Cluster genutzt werden, um diese zu verwalten oder eine Suche auf einzelne Cluster zu beschränken.\footnote{\cite[Vgl.][]{OrientDBCommands}} Der \texttt{JOIN} Befehl wurde durch \enquote{Links} ersetzt. Durch diese sind Verknüpfungen zwischen Daten einfacher zuzuordnen und technisch sind sie performanter umgesetzt als klassische \texttt{JOINS}. Verknüpfte Eigenschaften werden einfach durch eine Punkt-Notation angesprochen.\footnote{\cite[Vgl.][]{OrientDBSQL}} Ein Beispiel für die Punkt-Notation ist in Listing \ref{list:bsp_punkt_notation} zu sehen.

\begin{figure}[!htb]
    \begin{lstlisting}[caption=Beispiel der Punkt-Notation mit OrientDB Links, label=list:bsp_punkt_notation]
SELECT * FROM Employee WHERE city.country.name = 'Italy'
    \end{lstlisting}
\end{figure}

% Eventuell noch wie die LINKS aufgebaut sind? - Done

\subsubsection{Web Interface}
% http://orientdb.com/docs/2.2.x/Studio-Home-page.html

% Ausführen von Kommandos
% Dokumente bearbeiten
% Erstellen der Klassen, Clustern, Klasseneigenschaften, Indizes
% Visualisieren und Bearbeiten von Graph grafisch
% Datenbankeinstellungen (Datum und Zeit-format)
% Export der Datenbank
% technische Informationen über die Klassen und dazu die Cluster und Anzahl Records

Die Datenbank bietet ein Web Interface, das OrientDB Studio, für die einfache Bedienung. Mit diesem können \gls{SQL}-Abfragen ausgeführt und Dokumente im Browser bearbeitet werden. Wenn das Graph Modell genutzt wird, können sich hier auch die Graphen visualisiert werden. Außerdem lassen sich Verwaltungsaufgaben wie das Erstellen von Klassen und deren Eigenschaften, Clustern und Indizes über das Web Interface erledigen. Zudem werden weitere technische Einblicke in die Datenbank- und Clustergröße sowie die Anzahl der enthaltenen Datensätze gegeben. Datenbankeinstellungen wie die Standardsprache und das Datumsformat können hier ebenfalls angepasst werden.\footnote{\cite[Vgl.][]{OrientDBStudio}}

\subsubsection{Schnittstellen}
% http://orientdb.com/docs/2.2.x/Programming-Language-Bindings.html

% HTTP API, Java, JavaScript, Python, .NET

OrientDB bietet eine Reihe an Bibliotheken für den direkten Zugriff auf die Datenbank. Darunter Lösungen für Java, JavaScript, .NET und Python. Diese Bibliotheken bieten Funktionen für die Abfrage von Daten, das Einpflegen neuer oder veränderter Daten und weitere, um OrientDB spezifische Funktionen zu nutzen. Zusätzlich wird eine \gls{HTTP}-Schnittstelle angeboten. Einige Bibliotheken werden von freien Entwicklern programmiert und befinden sich deshalb nicht auf dem neusten Stand.\footnote{\cite[Vgl.][]{OrientDBBindings}}

\subsection{Datenquelle für Börsendaten}
% https://www.springerprofessional.de/data-collection-presentation-and-yahoo-finance/11237520?searchResult=1.yahoo%20finance&searchBackButton=true&fulltextView=true

% Yahoo finance
% Wie sehen die Daten aus, wo kommen sie her und wo werden sei gespeichert?
% Fehlen der Volumina für sehr alte Daten (aber eigentlich egal)
% bietet Aktien und Indexentwicklungen
% historische Daten
% direkter Download als CSV 
% auch als API
% auch aktuelle Kurse mit 0 bis 20 Minuten Verzögerung
% Diagramme über verschiedene Zeiträume
% Handelsvolumina
% Zusammensetzung von Indizes

% Handelsvolumen Definition überprüfen

Als Quelle für Börsendaten bietet sich \enquote{yahoo Finance} an. Auf der Plattform kann kostenlos die Kursentwicklung von vielen Einzelaktien und Indizes angesehen werden. Die historischen Kurswerte können direkt im \gls{CSV} Format heruntergeladen werden. Dabei werden Tagesöffnungs- und Schlusskurs sowie das Maximum, Minimum und das Handelsvolumen\footnote{Das Handelsvolumen ist die Anzahl der gehandelten Werte in einem bestimmten Zeitraum} zur Verfügung gestellt. Alternativ können diese Daten auch direkt über eine \gls{API} abgerufen werden. Zudem lassen sich auch die aktuelle Kurse teilweise in Echtzeit, größtenteils aber um einige Minuten verzögert, anzeigen lassen. Auf der Website bietet sich auch direkt die Möglichkeit der Darstellung als Graph und die Zusammensetzung von Indizes sowie deren Gewichtung lässt sich auch einsehen. Yahoo bezieht seine historischen Kursentwicklungen von den Finanzinformationsunternehmen Commodity Systems, Inc und Morningstar.\footnote{\cite[Vgl.][]{YahooFinanceSources}}

% Stooq

\clearpage