\section{Grundlagen}

\subsection{Aktienmarkt}

\subsubsection{Aktien}
% Erklärung von Aktienmarkt Indizes
% https://www.springerprofessional.de/equities-and-equity-indices/16861122?searchResult=2.equity%20indices&searchBackButton=true&fulltextView=true
% https://www.springerprofessional.de/geld-und-vermoegensanlage/19054466?searchResult=20.equity%20index&searchBackButton=true
% https://www.springerprofessional.de/aktien/10364288 - Aktien Angebot und Nachfrage

% Preis wird in Euro / Dollar angegeben

Aktien sind Unternehmensanteile. Investoren können sich über Aktien finanziell an Unternehmen beteiligen. Die öffentlichen Anteile können an einer Börse gehandelt werden. Mit dem Besitz einer Aktie bekommt der Aktionär ein Stimmrecht auf der Hauptversammlung und wird an Gewinnausschüttungen beteiligt. Der Preis oder auch Kurs einer Aktie schwankt in der Regel während der Öffnungszeiten der Börse. Er setzt sich aus Angebot und Nachfrage zusammen. Wichtige Werte sind die Tagesöffnungs- und Tagesschlusskurse. Diese geben den Kurs jeweils zur Öffnung und zum Schluss eines Tages an einer Börse an.

\subsubsection{Aktienindizes}
% DAX Berechnung alle 15 Sekunden - https://www.springerprofessional.de/equities-and-equity-indices/16861122?searchResult=2.equity%20indices&searchBackButton=true&fulltextView=true 
% DAX erfüllt Informationsfunktion - https://www.springerprofessional.de/geld-und-vermoegensanlage/19054466?searchResult=20.equity%20index&searchBackButton=true

% Nicht wie Aktien in Euro sondern in Punkten
% Warenkörbe von Aktien
% Bilden einen Markt oder eine Branche ab
Ein Aktienindex gibt die Wertentwicklung einer ganzen Gruppe von Aktien wider. Sein Wert wird, im Gegensatz zu Aktien, nicht in einer Währung wie Euro oder \gls{USD} angegeben, sondern in Punkten. Diese Punkte werden aus den Preisen der enthaltenen Werte und einer, je nach Art des Index unterschiedlichen, Gewichtung ausgerechnet. Daher kann in einen Index nicht direkt investiert werden. Er stellt lediglich eine Kenngröße für die Entwicklung einer Gruppe von Aktien dar. Für Anleger bietet ein Aktienindex Informationen über den allgemeinen Trend in einem Markt. Der \gls{DAX} vereint \gls{ZB} die Entwicklung der 30 umsatzstärksten Aktiengesellschaften Deutschlands. Er stellt damit eine übergreifende Kennziffer für die Entwicklung des Aktienmarkts in Deutschland. Andere Indizes decken ganze Kontinente oder einzelne Branchen ab. Da sich die unterliegenden Werte eines Index laufend während der Öffnungszeiten der Börse verändern, wird auch der Indexwert meistens im Laufe des Tages angepasst. Für den \gls{DAX} geschieht das alle 15 Sekunden.

\subsubsection{Kapitalmarktanomalien}
% Weekend Effect
% Anomalie, reicht zu sagen das sie existieren könnte das aber nicht bewiesen ist?



\subsection{NoSQL}

\subsection{OrientDB}

\subsubsection{Unterstützte Datenmodelle}

\subsubsection{Klassen und Schemata}

\subsubsection{Datenabfragesprache}

\subsubsection{Web Interface}

\subsubsection{Python Bibliothek}

\subsection{Datenquelle für Börsendaten}
% Wie sehen die Daten aus, wo kommen sie her und wo werden sei gespeichert?
% Fehlen der Volumina für sehr alte Daten (aber eigentlich egal)

\clearpage