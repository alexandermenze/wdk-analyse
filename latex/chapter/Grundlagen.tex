\section{Grundlagen}

\subsection{Aktienmarkt}

\subsubsection{Aktien}
% Erklärung von Aktienmarkt Indizes
% https://www.springerprofessional.de/equities-and-equity-indices/16861122?searchResult=2.equity%20indices&searchBackButton=true&fulltextView=true
% https://www.springerprofessional.de/geld-und-vermoegensanlage/19054466?searchResult=20.equity%20index&searchBackButton=true
% https://www.springerprofessional.de/aktien/10364288 - Aktien Angebot und Nachfrage

% Preis wird in Euro / Dollar angegeben

Aktien sind Unternehmensanteile. Investoren können sich über Aktien finanziell an Unternehmen beteiligen. Die öffentlichen Anteile können an einer Börse gehandelt werden. Mit dem Besitz einer Aktie bekommt der Aktionär ein Stimmrecht auf der Hauptversammlung und wird an Gewinnausschüttungen beteiligt. Der Preis oder auch Kurs einer Aktie schwankt in der Regel während der Öffnungszeiten der Börse. Er setzt sich aus Angebot und Nachfrage zusammen. Wichtige Werte sind die Tagesöffnungs- und Tagesschlusskurse. Diese geben den Kurs jeweils zur Öffnung und zum Schluss eines Tages an einer Börse an.

\subsubsection{Aktienindizes}
% DAX Berechnung alle 15 Sekunden - https://www.springerprofessional.de/equities-and-equity-indices/16861122?searchResult=2.equity%20indices&searchBackButton=true&fulltextView=true 
% DAX erfüllt Informationsfunktion - https://www.springerprofessional.de/geld-und-vermoegensanlage/19054466?searchResult=20.equity%20index&searchBackButton=true

% Nicht wie Aktien in Euro sondern in Punkten
% Warenkörbe von Aktien
% Bilden einen Markt oder eine Branche ab
Ein Aktienindex gibt die Wertentwicklung einer ganzen Gruppe von Aktien wider. Sein Wert wird, im Gegensatz zu Aktien, nicht in einer Währung wie Euro oder \gls{USD} angegeben, sondern in Punkten. Diese Punkte werden aus den Preisen der enthaltenen Werte und einer, je nach Art des Index unterschiedlichen, Gewichtung ausgerechnet. Daher kann in einen Index nicht direkt investiert werden. Er stellt lediglich eine Kenngröße für die Entwicklung einer Gruppe von Aktien dar. Für Anleger bietet ein Aktienindex Informationen über den allgemeinen Trend in einem Markt. Der \gls{DAX} vereint \gls{ZB} die Entwicklung der 30 umsatzstärksten Aktiengesellschaften Deutschlands. Er stellt damit eine übergreifende Kennziffer für die Entwicklung des Aktienmarkts in Deutschland. Andere Indizes decken ganze Kontinente oder einzelne Branchen ab. Da sich die unterliegenden Werte eines Index laufend während der Öffnungszeiten der Börse verändern, wird auch der Indexwert meistens im Laufe des Tages angepasst. Für den \gls{DAX} geschieht das alle 15 Sekunden.

\subsubsection{Kapitalmarktanomalien}
% https://www.springerprofessional.de/effiziente-maerkte/16235650?searchResult=3.kapitalmarktanomalie&searchBackButton=true
% Externer Faktor Wetter - http://arno.uvt.nl/show.cgi?fid=129516

% Weekend Effect
% Anomalie, reicht zu sagen das sie existieren könnte das aber nicht bewiesen ist?

% Fama Quelle

Die Effizienzmarkthypothese von Fama beschreibt, dass der Kapitalmarkt effizient ist, wenn alle relevanten Informationen allen Marktteilnehmern sofort zur Verfügung stehen und somit das Angebot und die Nachfrage der Marktteilnehmer direkt beeinflussen. Alle Marktteilnehmer handeln nach dieser Hypothese rational. Effizient bedeutet dann, dass bei der Veröffentlichung neuer Informationen über ein Unternehmen sich die Preise einer Aktie sofort anpassen um den korrekten Wert widerzuspiegeln. Diese Theorie beschreibt einen perfekten Markt wie er in der Realität nicht existiert. Untersuchungen von ... zeigten, dass neben zufälligen Preisentwicklungen auch externe Faktoren, wie das Wetter, einen Einfluss haben und auch periodische Anomalien auftreten, die der strikten Hypothese widersprechen. Darunter gibt es jährlich, saisonal, und wöchentlich auftretende allgemeine Schwankungen. Im Rahmen dieser Arbeit soll sich vor allem auf den \enquote{Weekend Effect} konzentriert werden. Dieser besagt, dass die Kurse am Montag im Vergleich zu den anderen Wochentagen unterdurchschnittlich ausfallen. Zu dieser Erscheinung gibt es viele Untersuchungen und Vermutungen aber keine wissenschaftlich belegte Erklärung.

\subsection{NoSQL}
% https://www.springerprofessional.de/nosql-databases/19181680?searchResult=1.nosql&searchBackButton=true&fulltextView=true
% https://www.springerprofessional.de/a-comparative-study-of-nosql-databases/18759118?searchResult=10.nosql&searchBackButton=true&fulltextView=true

% Entwicklung von relationalen Datenbanken früher für alles (waren auch praktisch für shop Systeme mit transaktionaler Sicherheit, single source of truth)
% Massensysteme die weltweit verteilt sind haben Probleme aufgezeigt (Skalierung und Performance wegen ACID)
% Ermöglichen eine horizontale Skalierung (Auf mehrere Systeme)
% Gerade bei Big Data und Massendaten die unterschiedlich strukturiert sind ist es schwierig ein gemeinsames Datenmodell zu finden
% Typen: Key-Value Datastore, Column-oriented Data Store, Document Data Store, Graph Data Store

Angetrieben durch die hohen Anforderungen an Skalierung und Performance entwickelten sich in den späten 80er Jahren die sogenannten NoSQL Datenbanken. Markant ist vor allem die Nutzung eines nicht-relationalen Datenbankmodells. Bei diesen muss im Vorhinein kein festes Schema festgelegt werden, wodurch die Administrationsaufgaben minimiert werden. Außerdem unterstützen viele dieser Datenbanken eine Replikation auf mehrere kleine Instanzen und damit auch eine erhöhte Ausfallsicherheit und verbesserte Lastenverteilung. Um diese Ziele zu erreichen verzichten NoSQL Datenbanken häufig auf die transaktionale Sicherheit die bei relationalen Datenbanken üblich ist. Da dies nicht die optimale Lösung für jedes System ist, stellen NoSQL Datenbanken keinen Ersatz oder eine Verbesserung dar sondern eine Alternative für spezielle Anwendungsfälle.

Die vier am häufigsten vertretenen Datenmodelle sind \enquote{Key-Value}, \enquote{Column-oriented}, \enquote{Document} und \enquote{Graph}. Der Key-Value Speicher funktioniert ähnlich wie ein Hash-Table. Jedem Eintrag wird ein eindeutiger Schlüssel zugeteilt. Dabei kann der Wert dahinter ein \gls{JSON} Dokument sein. Bei dem Column-oriented Datenmodell werden die Daten in Tabellen gespeichert, jedoch werden die Daten in einem anderen Format als bei einer relationalen Datenbank auf der Festplatte abgespeichert. Dieses soll vor allem die Lese-Zugriffe auf den Speicher deutlich reduzieren. In den dokumentenbasierten NoSQL Datenbanken können unstrukturierte Daten ohne festes Schema gespeichert werden. Jeder Datensatz ist dabei ein eigenes Dokument. Eine Graph-Datenbank erlaubt das Speichern von großen Datenmengen und Beziehungen unter diesen Daten.

\subsection{OrientDB}

\subsubsection{Unterstützte Datenmodelle}

\subsubsection{Klassen und Schemata}

\subsubsection{Datenabfragesprache}

\subsubsection{Web Interface}

\subsubsection{Python Bibliothek}

\subsection{Datenquelle für Börsendaten}
% Wie sehen die Daten aus, wo kommen sie her und wo werden sei gespeichert?
% Fehlen der Volumina für sehr alte Daten (aber eigentlich egal)

\clearpage