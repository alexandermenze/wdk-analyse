\section{Einleitung}
% Aktive Aktionäre sind immer auf der Suche nach Möglichkeiten den Markt zu schlagen
% NoSQL und Docker kurz ansprechen
% Warum sind Datenanalysen heute so toll?
% Mit steigender Verfügbarkeit an Börsendaten können immer mehr Analysen gemacht werden (wegen Internet)
% Durch die Nutzung einer starken NoSQL Datenbank können vor allem große Datenmengen schnell verarbeitet werden

% Mythos das man Montag immer an der Börse Geld verliert (Quelle suchen). Das will man weiter untersuchen
% Dafür sollen historische Daten untersucht werden und eine Datenbank verwendet werden.

In seinem Buch \enquote{Nichtlineare Zeitreihenanalyse als neue Methode für Eventstudien} beschreibt der Autor Wagner verschiedene periodische Schwankungen in den Aktienmärkten. Unter anderem den \enquote{Weekend Effect}. Dieser besagt, dass die Preisverläufe von Aktien Montags im Durchschnitt niedriger liegen als an den restlichen Wochentagen.\footnote{\cite[Vgl.][17]{Wagner2019}} Dafür liefert der Autor Coenen verschieden mögliche Gründe. Einer könnte sein, dass die Unternehmen Nachrichten die sich negativ auf den Aktienkurs auswirken überwiegend am Wochenende veröffentlicht würden um keine Panikreaktion auszulösen. Dagegen spricht allerdings das sich diese Nachrichten dann länger als nur über den Montag ziehen müssten. Eine andere Theorie ist, dass durch den reduzierten Handel zum Wochenstart die Preise sinken. Allerdings spreche dagegen, dass dieser Effekt nicht an Feiertagen auftreten solle.\footnote{\cite[Vgl.][8]{Coenen2020}}

\clearpage